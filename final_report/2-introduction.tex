Optical communication systems provide the means for high throughput high speed communication and are used in a variety of applications from the transatlantic fiber cable to inter-data-center communication. Fast and reliable communication is a key requirement in such applications and is hindered by phenomena such as chromatic dispersion and non-linear photo-detection \autocite{8433895}.
\\

In the past decade, machine learning has been used in an increasing number of classification, regression and forecasting solutions, performing faster and with higher accuracy than the best traditional algorithms. There has been a high interest in moving Neural Networks away from high cost, low efficiency CPUs and GPUs towards Field Programmable Gate Arrays (FPGAs) and Application-Specific Integrated Circuit (ASICs) \autocite{7929192}. Neural network implementations on FPGA allow an order of magnitude improvement in power efficiency, making them cheaper and even suitable for embedded Internet of Things (IoT) applications \autocite{7799795,8954866,8469659,8330546,8693488}. Furthermore, the flexibility of FPGAs enables building large scale neural network \autocite{8823487,7045812}, enabling even higher performance than current top of the line GPUs \autocite{8702332,8412552}. 
\\

This project tries to recreate and build on the work done by Boris Karanov et al. in \autocite{8433895}. The proposed neural network architecture was recreated and tweaked slightly to further optimize performance. The goal is to realize the proposed model in \autocite{8433895} on an FPGA for real-time implementation in an optical fiber link. In addition to this, concepts such as quantised neural networks were explored to improve performance of the hardware implementation. The concept is that, instead of using manually engineered modulation formats, the encoding and decoding neural networks will be able to learn a more optimal modulation scheme that is able to compensate for chromatic dispersion and non-linear photodiode detection.
\\

This report is structured so that most of the theoretical bases are stated in \S \ref{sec:theory}. Details on how the proposed solution was designed and realized are given in \S \ref{sec:methods}. The results of this project as well as their discussion is provided in \S \ref{sec:results} and finally a conclusion and evaluation is given in \S \ref{sec:conclusion}.
