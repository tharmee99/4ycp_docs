% !TeX root = proposal.tex

\iffalse
Think of the introduction as a development of the context for the reader.

Peruse the literature quickly. 
Highlight the historical progression of the theory and work that has been done, citing the literature as you go. 

Use the citation mechanisms in the editor you are using. 
After the reader has some idea of what the historical background is, mention what is missing in it that your proposal is going to fill.


Optical Fibres are a staple medium for fast communication systems all over the globe. They allow for light speed transfer of bits between two stations which has numerous applications in today's society. However, they are not perfect. They can suffer from a few issues such as Additive White Gaussian Noise (AWGN) and Shot Noise from Photo-Diodes as well as chromatic dispersion. This project will be addressing a method to mitigate the adverse of effects of non-linearities in the optical fibre cables. In particular, the effect of implementing Neural Networks (NNs) in both the transmitter and receiver of a communication system to enable "end to end deep learning" will be observed and improved upon with the use of Field Programmable Gate Arrays (FPGAs).  

\fi

In the past decade, machine learning has been used in an increasing number of classification, regression and forecasting problem solutions, performing faster and with higher accuracy than some best traditional alogrithms. There has been a high interest in moving Neural Networks (NNs) away from high cost, low efficiency CPUs and GPUs towards Field Programmable Gate Arrays (FPGAs) and Application-Specific Integrated Circuit (ASICs) \autocite{7929192}. NN implementations on FPGA shows a margin of magnitute higher power efficiency that could be cheaper and even suitable for embedded Internet of Things (IoT) applications \autocite{7799795,8954866,8469659,8330546,8693488}. Furthermore, flexibility of FPGA enables building large scale NN \autocite{8823487,7045812}, enabling even higher performance than current top of the line GPUs \autocite{8702332,8412552}. 
\\
\\
Optical communication systems provide a high throughput high speed communication system and are used in a variety of applications from the transatlantic fiber cable to inter datacenter communication. Fast and reliable communication is a key requirement in such applications and is hindered by phenomena such as chromatic dispersion and non-linear photodiode detection \autocite{8433895}. This project will attempt to combine the performance of machine learning techniques with the flexibility and scalability of FPGAs in order to mitigate non-linear distortions and noise introduced in optical communication systems and thereby increase the maximum data throughput that is achievable by the communication system. 
\\
\\
As oppposed to manually engineering the modulation scheme for a communication system, a NN as the modulator can be treated as a black box and learn the optimal modulation scheme for each instance of a communication channel. This not-only ensures a optimal modulation scheme for a communication system but is also able to adapt to each unique implementation of an optical communication system. Likewise, a NN at the receiving end can be treated as a black box equalizer and demodulator  as opposed to manually implementing digital backpropagation or other compensation techniques. The NN will be implemented on FPGAs for optimal computing speeds as well as power efficiency and should be able to deal with the high throughput that optical communication systems are subject to. 