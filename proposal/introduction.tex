% !TeX root = proposal.tex

\iffalse
Think of the introduction as a development of the context for the reader.

Peruse the literature quickly. 
Highlight the historical progression of the theory and work that has been done, citing the literature as you go. 

Use the citation mechanisms in the editor you are using. 
After the reader has some idea of what the historical background is, mention what is missing in it that your proposal is going to fill.
\fi

Optical Fibres are a staple medium for fast communication systems all over the globe. They allow for light speed transfer of bits between two stations which has numerous applications in today's society. However, they are not perfect. They can suffer from a few issues such as Additive White Gaussian Noise (AWGN) and Shot Noise from Photo-Diodes as well as chromatic dispersion. This project will be addressing a method to mitigate the adverse effects of the non-linearities in the optical fibre cables. In particular, the effect of implementing neural networks in both the transmitter and receiver of a communication system to enable "end to end deep learning" will be observed and improved upon with the use of Field Programmable Gate Arrays (FPGAs). Different neural network architectures will be theorised and tested on FPGAs for communication systems. We aim to have real-time conversion of modulation schemes at a rate of at least 112 Gbps for a transmission medium length of approximately 100 km. These figures are very typical for most data centres and the purpose of this project is for this system to have a large capacity for a low cost, this is done through reducing computational cost because dedicated hardware is being implemented.

\newline \newline \textit{other way I was thinking to extend this was to add what other major papers were lacking and how we make up for this} 